\section{Введение}

\textbf{Python} -- высокоуровневый язык программирования общего назначения с динамической строгой типизацией и автоматическим управлением паматью, ориентированный на повышение производительности разработчика, читаемости кода и его качества, а также на обеспeчение переносимости написанных на нём программ.

Язык является полностью объектно-ориентированным в том плане, что всё является объектами. Необычной особенностью языка является выделение блоков кода пробельными отступами.
Недостатками языка являются зачастую более низкая скорость работы и более высокое потребление памяти написанных на нём программ по сравнению с аналогичным кодом, написанным на компилируемых языках, таких как C или C++.

Стандартная библиотека включает большой набор полезных переносимых функций, начиная от функционала для работы с текстом и заканчивая средствами для написания сетевых приложений. Дополнительные возможности, такие как математическое моделирование, работа с оборудованием, написание веб-приложений или разработка игр, могут реализовываться посредством обширного количества сторонних библиотек, а также интеграцией библиотек, написанных на Си или C++, при этом и сам интерпретатор Python может интегрироваться в проекты, написанные на этих языках. Существует и специализированный репозиторий программного обеспечения, написанного на Python, — PyPI. Данный репозиторий предоставляет средства для простой установки пакетов в операционную систему и стал стандартом де-факто для Python. По состоянию на 2019 год в нём содержалось более 175 тысяч пакетов.

Python стал одним из самых популярных языков, он используется в анализе данных, машинном обучении, DevOps и веб-разработке, а также в других сферах, включая разработку игр. За счёт читабельности, простого синтаксиса и отсутствия необходимости в компиляции язык хорошо подходит для обучения программированию, позволяя концентрироваться на изучении алгоритмов, концептов и парадигм. Отладка же и экспериментирование в значительной степени облегчаются тем фактом, что язык является интерпретируемым. Применяется язык многими крупными компаниями, такими как Google или Facebook. По состоянию на октябрь 2021 года Python занимает первое место в рейтинге TIOBE популярности языков программирования с показателем 11,27\%. «Языком года» по версии TIOBE Python объявлялся в 2007, 2010, 2018 и 2020 годах.

\subsection{Основные редакторы кода}

В рамках данной работы предлагается работать с Python в трёх редакторах кода: VS Code, Colab и Pycharm.

\subsubsection{Visual Studio Code}

\textbf{Visual Studio Code (VS Code)} — редактор исходного кода, разработанный Microsoft для Windows, Linux и macOS. Позиционируется как «лёгкий» редактор кода для кроссплатформенной разработки веб- и облачных приложений. Включает в себя отладчик, инструменты для работы с Git, подсветку синтаксиса, IntelliSense и средства для рефакторинга. Имеет широкие возможности для кастомизации: пользовательские темы, сочетания клавиш и файлы конфигурации. Распространяется бесплатно, разрабатывается как программное обеспечение с открытым исходным кодом.

Данный редактор поддерживает подсветку синтаксиса и отладку Python (при установленном плагине Python).

Для скачивания библиотек необходимо написать в терминале:

\begin{center}
	pip install название\_библиотеки
\end{center}

Скачать редактор можно по ссылке: \url{https://code.visualstudio.com}

\subsubsection{Colab}

\textbf{Colab} -- это бесплатная интерактивная облачная среда для работы с кодом от Google. В основе лежит блокнот Jupyter для работы на Python, только с базой на Google Диске, а не на компьютере. Главная особенностью являются бесплатные мощные графические процессоры GPU и TPU, благодаря которым можно заниматься не только базовой аналитикой данных, но и более сложными исследованиями в области машинного обучения.

Еще одно достоинство Colab — интеграция с GitHub. Он открывает доступ к любому хранилищу, если ему предоставить профиль на сервисе.

Для скачивания библиотек необходимо написать в блокноте:

\begin{center}
	\%pip install название\_библиотеки
\end{center}

Воспользоваться редакторов можно по ссылке: \url{https://colab.research.google.com} \\ (необходимо иметь Google-аккаунт).


\subsubsection{Pycharm}

\textbf{Pycharm} -- это среда разработки от компании JetBrains, которая специализируется на создании продуктов для программистов, в том числе всяких IDE \textit(Integrated Development Environment). Предоставляет средства для анализа кода, графический отладчик, инструмент для запуска юнит-тестов и поддерживает веб-разработку на Django. Pycharm совместим с Windows, macOS, Linux; он также имеет несколько вариантов лицензий, которые отличаются функциональностью, стоимостью и условиями использования. \textit{Pycharm Community Edition} -- бесплатная версия данного продукта.

Для скачивания библиотек необходимо написать в терминале:

\begin{center}
	pip install название\_библиотеки
\end{center}

Скачать редактор можно по ссылке: \url{https://www.jetbrains.com/ru-ru/pycharm/download/#section=windows}

\newpage
