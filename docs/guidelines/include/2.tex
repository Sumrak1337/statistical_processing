\newpage
\addcontentsline{toc}{section}{Элементы анализа данных}
\section*{Элементы анализа данных}
% https://kpfu.ru/portal/docs/F_640167679/Bespalov.Matmetody_2020_1.pdf
Математическая статистика -- это раздел математики, посвященный методам сбора, анализа и обработки статистических данных для научных и практических целей.

Статистические данные -- такие данный, которые получены в результате обследования большого числа объектов или явлений, то есть, математическая статистика имеет дело с массовыми явлениями.

Статистический анализ данных включает в себя:

\begin{itemize}
	\item \textit{Описательная статистика (Descriptive Statistics)}
	
	Сюда входят методы описания статистических данных, представления их в форме таблиц и распределений и пр.
	
	\item \textit{Индуктивная статистика (Infernal Statistics)}
	
	По-другому это может называться как \textit{"аналитическая статистика"} или \textit{"теория статистических выводов"}. В основном здесь реализуются обработка данных, которые были получены в ходе эксперимента, и формулировка выводов, имеющих прикладное значение для конкретной области человеческой деятельности. Теория статистических выводов базируется на математическом аппарате теории вероятности.
\end{itemize}

Предметом изучения в статистике являются изменяющиеся (варьирующие) признаки, которые называются статистическими признаками.

Наличие общего признака является основой для образования статистической совокупности. То есть, статистическая совокупность -- это результаты описания или измерения общих признаков объектов исследования.

Признаки и переменные -- это измеряемые явления. Значения признака определяются при помощи специальных шкал наблюдения. 

Данные -- результаты некоторого количества измерений какой-либо переменной (признака) или переменных (признаков) \textit{(вес, длина тела, пол, температура...)}.

Данные бывают:

\begin{itemize}
	\item Категориальные \textit{(Качественные)}
	\begin{itemize}
		\item Номинальные \textit{(nominal)}
		
		Взаимоисключающие и неупорядоченные категории (нельзя выстроить в последовательность);
		
		\item Порядковые \textit{(ordinal)}
		
		Взаимоисключающие и упорядоченные категории (могут быть упорядочены, но размер интервалов на шкале может быть не одинаков);
	\end{itemize}
	\item Количественные \textit{(Числовые)}
	\begin{itemize}
		\item Дискретные \textit{(discrete)}
		
		Как правило, целочисленные значения, типичные для счёта;
		
		\item Непрерывные \textit{(continuous)}
		
		Любые значения в определенном интервале.
	\end{itemize}
\end{itemize}

Три основных концепции в анализе данных:

\begin{enumerate}
	\item Распределение переменной и его описание;
	\item Распределение выборочных средних и связь её с распределением переменной;
	\item Статистика Критерия.
\end{enumerate}

\addcontentsline{toc}{subsection}{Распределение переменной}
\subsection*{Распределение переменной}

Частотное распределение переменной \textit{(frequency distribution)} -- это соответствие между значениями переменной и их вероятностями (или количеством таких значений в выборке).

Вариационный ряд -- ряд, в котором сопоставлены (по степени возрастания или убывания) варианты и соответствующие им частоты.

Частота (\colorbox[rgb]{0.95, 0.95, 0.95}{$f_i$}) -- число, показывающее сколько раз повторяется значение (\colorbox[rgb]{0.95, 0.95, 0.95}{$x_i$}) признака в вариационном ряду.

Частостью или относительной частотой (\colorbox[rgb]{0.95, 0.95, 0.95}{$w_i$}) называется отношение частоты к объему выборки (\colorbox[rgb]{0.95, 0.95, 0.95}{n}):

\begin{equation}
	w_i = \frac{f_i}{n}
\end{equation}

Наиболее популярными и употребительными графиками для изображения вариационных рядов, то есть соотношений между значениями признака и соответствующими частотами или относительными частотами, являются \textit{гистограмма} и \textit{полигон частот}.

Гистограмма -- графическое представление частотного распределения, разбитого по интервалам, где высота столбика отражает частоту.
% todo: картинка

Полигон частот используют для дискретных рядов. На оси абсцисс откладываются значения аргумента, а на оси ординат -- значения частот или относительных частот.
% todo: картинка

Генеральная совокупность -- множество всех объектов в рассматриваемой конкретной задаче.

Выборочная совокупность или выборка -- часть генеральной совокупности, которая охватывается тем или иным экспериментом (наблюдением, опросом). Чтобы свойства выборки отражали свойства генеральной совокупности \textit{(репрезентативная выборка)}, сама выборка должна быть случайной, то есть, все объекты в генеральной совокупности должны иметь одинаковые шансы в неё, и попадание в выборку одного объекта никак не должно влиять на попадание другого.

Основные характеристики частотного распределения переменной:

\addcontentsline{toc}{subsubsection}{"Середина" распределения}
\subsubsection*{"Середина" \hspace*{0.1cm} распределения}

\begin{itemize}
	\item Среднее значение \textit{(mean)}
	\item Медиана \textit{(median)}
	\item Мода \textit{(mode)}
\end{itemize}

Это величины могут служить оценками генеральной совокупности, а в выборке -- это наиболее эффективная и несмещённая оценка.

\addcontentsline{toc}{paragraph}{Среднее значение}
\paragraph*{Среднее значение} \mbox{} \\

Среднее значение -- сумма всех значений переменной, делённая на количество значений. 
% todo: картинка

Среднее для выборки:
\begin{equation}
	\overline{X} = \frac{\sum\limits_i X_i}{n}
\end{equation}

Среднее для генеральной совокупности:
\begin{equation}
	\overline{\mu} = \frac{\sum X}{N}
\end{equation}

Для вычисления среднего используются все значения набора данных, само значение определяется математически выполнимым алгебраическим выражением, а также известно выборочное распределение. К недостаткам стоит отнести сильную зависимость от выбросов и асиммитричных данных.

\addcontentsline{toc}{paragraph}{Медиана}
\paragraph*{Медиана} \mbox{} \\

Медиана - значение, которое делит распределение пополам. То есть, половина значений больше медианы, другая половина -- не больше.
%todo: картинка

Если распределение не симметричное, медиана лучше характеризует центр распределения. Она содержит меньше информации, чем среднее, но зато она не чувствительна к выбросам и асиметричным данным и может применяться даже в случае, если не для всех значений измерения точные. Стоит отметить, что значение этой характеристики не определяется алгебраически, что может затруднить дальнейшее описание.

Распределение также можно поделить не на две части, а на четыре \textit{(квартили)}, восемь \textit{(октили)}, сто \textit{(процентили)}, N \textit{(квантили)}.

Квартили \textit{(quartiles)} делят распределение на четыре части так, что в каждой из них оказывается поровну значений.

\begin{itemize}
	\item 1-ая квартиль == 25\% процентиль;
	\item 2-ая квартиль == медиана;
	\item 3-ая квартиль == 75\% процентиль.
\end{itemize}

Интерквартильный размах -- разница между третьей и первой квартилями.

\addcontentsline{toc}{paragraph}{Мода}
\paragraph*{Мода} \mbox{} \\

Мода -- наиболее часто встречающееся значение.
% todo: картинка

Данная характеристика легко определяется для категориальных данных, но она игнорирует большую часть информации, не определяется алгебраически, а также ничего не известно про выборочное распределение.

\addcontentsline{toc}{subsubsection}{"Ширина" распределения}
\subsubsection*{"Ширина" \hspace{0.1cm} распределения}

\begin{itemize}
	\item Размах \textit{(range)}
	\item Стандартное отклонение \textit{(standart deviation)}
	\item Дисперсия \textit{(variance)}
\end{itemize}

\addcontentsline{toc}{paragraph}{Размах}
\paragraph*{Размах} \mbox{} \\

Размах -- разность между максимальным и минимальным значениями:

\begin{equation}
	range = X_n - X_1
\end{equation}

Размах хорош тем, что легко считается. Плох тем, что зависит только от 2-х точек из распределения. Недооценивает истинный размах генеральной совокупности. Для описания "Ширины" распределения помимо размаха стоит привести ещё какую-нибудь характеристику разброса.

\addcontentsline{toc}{paragraph}{Стандартное отклонение}
\paragraph*{Стандартное отклонение} \mbox{} \\

