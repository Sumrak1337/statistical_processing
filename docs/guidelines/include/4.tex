\section{Обзор библиотек Python}

\subsection{Pandas}

\begin{enumerate}
	\item Индексирование, манипулирование, переименование, сортировка, объединение фрейма данных;
	\item Обновить, добавить, удалить столбцы из фрейма данных;
	\item Восстановить недостающие файлы, обработать недостающие данные или NAN;
	\item Построить гистограмму или прямоугольную диаграмму.
\end{enumerate}

\subsection{NumPy}

\begin{enumerate}
	\item Основные операции с массивами: добавление, умножение, срез, выравнивание, изменение формы, индексирование массивов;
	\item Расширенные операции с массивами: стековые массивы, разбиение на секции, широковещательные массивы;
	\item Работа с DateTime или линейной алгеброй;
	\item Основные нарезки и расширенное индексирование в NumPy Python.
\end{enumerate}

\subsection{SciPy}

\begin{enumerate}
	\item Математические, научные, инженерные вычисления;	
	\item Процедуры численной интеграции и оптимизации;
	\item Поиск минимумов и максимумов функций;
	\item Вычисление интегралов функции;
	\item Поддержка специальных функций;
	\item Работа с генетическими алгоритмами;
	\item Решение обыкновенных дифференциальных уравнений.
\end{enumerate}

\subsection{Matplotlib}

\begin{enumerate}
	\item Линейные диаграммы;
	\item Точечные диаграммы;
	\item Диаграммы с областями;
	\item Столбцовые диаграммы и гистограммы;
	\item Круговые диаграммы;
	\item Диаграммы «стебель-листья»;
	\item Контурные графики;
	\item Поля векторов;
	\item Спектрограммы.
\end{enumerate}

\subsection{Seaborn}

\begin{enumerate}
	\item Определять отношения между несколькими переменными (корреляция);
	\item Соблюдать качественные переменные для агрегированных статистических данных;
	\item Анализировать одномерные или двумерные распределения и сравнивать их между различными подмножествами данных;
	\item Построить модели линейной регрессии для зависимых переменных;	
	\item Обеспечить многоуровневые абстракции, многосюжетные сетки.
\end{enumerate}

\newpage