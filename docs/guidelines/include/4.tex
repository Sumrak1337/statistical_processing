\section{Обзор библиотек Python для проведения анализа данных и моделирования}

Уже довольно давно Python очаровывает учёных, занимающихся данными. Чем больше идёт взаимодействие с ресурсами, литературой, курсами, тренингами и людьми в науке о данных, тем более глубокие знания о Python приобретаются.

Специалисты в области Data Science точно знают о библиотеках Python, которые можно использовать в науке о данных, но когда в интервью просят назвать их или указать их функции, они часто могут не вспомнить даже и 5 библиотек.

Ниже представлен список и краткое изложение библиотек Python, которые помогают в области Data Science.

\subsection{Pandas}

\textit{Pandas} -- это пакет Python с открытым исходным кодом, который предоставляет высокоэффективные, просты в использовании структуры данных и инструменты анализа. Pandas -- это инструмент, предназначенный для быстрой и простой обработки данных, чтения, агрегирования и визуализации.

Pandas берёт данные из файлов CSV, TSV или базы данных SQL и создает объект Python со строками и столбцами, который называется \textit{фреймом данных}. Фрейм данных очень похож на табоицу в статистическом программном обеспечении (например, в Excel или SPSS).

Что можно делать с помощью Pandas?

\begin{enumerate}
	\item Индексирование, манипулирование, переименование, сортировка, объединение фрейма данных;
	\item Обновить, добавить, удалить столбцы из фрейма данных;
	\item Восстановить недостающие файлы, обработать недостающие данные или NAN;
	\item Построить гистограмму или прямоугольную диаграмму.
\end{enumerate}

Это делает Pandas фундаментальной библиотекой в изучении Python для Data Science.

\subsection{NumPy}

\textit{NumPy} -- один из самых основных пакетов в Python, универсальный пакет для обработки массивов. Он предоставляет высокопроизводительные объекты многомерных массивов и инструменты для работы с ними. NumPy -- эффективный контейнер универсальных многомерных данных. Основной объект NumPy -- это однородный многомерный массив.

NumPy используется для обработки массивов, в которых хранятся значения одного и того же типа данных. Этот пакет облегчает математические операции над массивами и их векторизацию, что значительно повышает производительно, и, соответственно, ускоряет время выполнения.

Что можно делать с помощью NumPy?

\begin{enumerate}
	\item Основные операции с массивами: добавление, умножение, срез, выравнивание, изменение формы, индексирование массивов;
	\item Расширенные операции с массивами: стековые массивы, разбиение на секции, широковещательные массивы;
	\item Работа с DateTime или линейной алгеброй;
	\item Основные нарезки и расширенное индексирование в NumPy Python.
\end{enumerate}

\subsection{SciPy}

Библиотека \textit{SciPy} является одним из ключевых пакетов, которые составляют стек SciPy. Он основывается на обхекте массива NumPy и является частью стека, который включает в себя такие инструменты, как Matplotlib, Pandas и SymPy с дополнительными инструментами.

Библиотека Scipy содержит модули для эффективных математических процедур, таких как линейная алгебра, интерполяция, оптимизация, интеграция и статистика. Основной функционал SciPy построен на NumPy и его массивах.

SciPy использует массивы в качестве базовой структуры данных. Он имеет различные модули для выполнения общих задач научного программирования, таких как линейная алгреба, интеграция, математический анализ, обыкновенные дифференциальные уравнения и обработка сигналов.

Что можно делать с помощью SciPy?

\begin{enumerate}
	\item Математические, научные, инженерные вычисления;	
	\item Процедуры численной интеграции и оптимизации;
	\item Поиск минимумов и максимумов функций;
	\item Вычисление интегралов функции;
	\item Поддержка специальных функций;
	\item Работа с генетическими алгоритмами;
	\item Решение обыкновенных дифференциальных уравнений.
\end{enumerate}

\subsection{Matplotlib}

\textit{Matplotlib} -- основная библиотека Python, предназначенная для визуализации данных и предоставляющая API для встраивания графиков в приложения. Очень напоминает MATLAB, встроенный в язык программирования Python.

Что можно делать с помощью Matplotlib?

\begin{enumerate}
	\item Линейные диаграммы;
	\item Точечные диаграммы;
	\item Диаграммы с областями;
	\item Столбцовые диаграммы и гистограммы;
	\item Круговые диаграммы;
	\item Диаграммы «стебель-листья»;
	\item Контурные графики;
	\item Поля векторов;
	\item Спектрограммы.
\end{enumerate}

\subsection{Seaborn}

\textit{Seaborn} -- это библиотека визуализации данных на основе Matplotlib, предоставляющая высокоуровенвый интерфейс для изображения интересных и информативных статистических графиков. Проще говоря, Seaboen -- это расширение Matplotlib с дополнительными возможностями. Разница между ними состоит в том, что Matplotlib используется для основного построения столбцовых, круговых, линейных, точечных диаграмм, в то время как Seaborn предоставляет множество шаблонов визуализации с меньши количеством синтаксических правил.

Что можно делать с помощью Seaborn?

\begin{enumerate}
	\item Определять отношения между несколькими переменными (корреляция);
	\item Соблюдать качественные переменные для агрегированных статистических данных;
	\item Анализировать одномерные или двумерные распределения и сравнивать их между различными подмножествами данных;
	\item Построить модели линейной регрессии для зависимых переменных;	
	\item Обеспечить многоуровневые абстракции, многосюжетные сетки.
\end{enumerate}

\subsection{Scikit Learn}

\textit{Scikit Learn} -- представляет собой надёжную библиотеку машинного обучения для Python. он включает в себя алгоритмы Machine Learning, такие как SVM, Random Forests, K-Means кластеризацию, спектральную кластеризацию, сдвиг среднего значения, перекрёстную проверку и другие. 

Scikit Learn предоставляет ряд контролируемых и неконтролируемых алгоритмов обучения через согласованный интерфейс в Python. Данный пакет будет руководством для того, чтобы модели контролируемого обучения, например, Naive Bayes, группировали непомеченные данные, такие как K-Means.

Что можно сделать с помощью Scikit Learn?

\begin{enumerate}
	\item Классификация: обнаружение спама, распознавание изображений;
	\item Кластеризация: воздействия лекарственных препаратов, цена акций;
	\item Регрессия: сегментация клиентов, группировка результатов эксперимента;
	\item Уменьшение размерности: визуализация, повышенная эффективность;
	\item Выбор модели: повышенная точность благодаря настройке параметров;
	\item предварительная обработка: подготовка входных данных в виде текста для обработки с помощью алгоритмов машинного обучения.
\end{enumerate}

\subsection{TensorFlow}

\textit{TensorFlow} -- это библиотека искуственного интеллекта (AI), которая помогает разработчикам создавать крупномасштабные нейронные сети со многими слоями, используя графики потоков данных. TensorFlow также облегчает построение моделей глубокого обучения, продвигает современную технологию ML/AI и позволяет легко развертывать приложения на базе ML.

TensorFlow достаточно эффективен, когда дело доходит до классификации, восприятия, понимания, обнаружения, прогнозирования и создания данных.

Что можно делать с помощью TensorFlow?

\begin{enumerate}
	\item Распознование голоса/звука, интернет вещей, автомобильная промышленность, безопасность, UX/UI, телекоммуникации;
	\item Анализ настроений -- в основном для CRM или CX;
	\item Текстовые приложения -- обнаружение угроз, Google Translate, Gmail Smart Reply;
	\item Распознавание лиц -- Facebook's Deep Face, Photo tagging, Smart Unlock;
	\item Временные ряды -- рекомендации от Amazon, Google и Netflix;
	\item Обнаружение видео -- обнаружение движения, обнаружение угроз в реальном времени в играх, безопасности, аэропортах.
\end{enumerate}

\subsection{Keras}

\textit{Keras} -- это высокоуровневый API TensorFlow для создания и обучения кода глубоких нейронных сетей. Это библиотека нейронных сетей с открытым исходным кодом на Python. С Keras статистическое моделирование, работа с изображениями и текстом намного легче с упрощённым для глубокого обучения. Keras создан для Python и делает его более удобным, модульным и компонуемым, чем TensorFlow.

Что можно делать с помощью Keras?

\begin{enumerate}
	\item Определить процентную точность;
	\item Функция вычисления потерь;
	\item Создать пользовательские функциональные слои;
	\item Встроенные функции обработки данных и изображений;
	\item Функции с повторяющимися блоками кода: глубиной 20, 50, 100 слоев.
\end{enumerate}

\subsection{Statsmodels}

\textit{Statsmodels} -- это универсальный пакет Python, который обеспечивает простые вычисления для описательной статистики и оценки и формирования статистических моделей. Данный пакет упрощает проведение статистических тестов и исследование статистических данных.

Что можно делать с помощью Statsmodels?

\begin{enumerate}
	\item Линейная регрессия;
	\item Корреляция;
	\item Метод наименьшего квадрата (OLS);
	\item Анализ выживания;
	\item Обобщённые линейные модели и байесовская модель;
	\item Однофакторный и двухфакторный анализ, проверка гипотез.
\end{enumerate}

\subsection{Plotly}

\textit{Plotly} -- это графическая библиотека для Python. Пользователи могут импортировать, копировать, вставлять или передавать данные, которые должны быть проанализированы и визуализированы. Данный пакет можно использовать, когда необходимо создовать, отображать и обновлять фигуры, наводить курсор на текст для получения подробной информации. PLotly также имеет дополнительную функцию отправки данных на облачные серверы.

Что можно делать с помощью Plotly?

\begin{enumerate}
	\item Основные диаграммы:
	\begin{itemize}
		\item Линейные;
		\item Круговые;
		\item Точечные;
		\item Пузырьковые;
		\item Графики Ганта;
		\item Санбёрст;
		\item Древовидые;
		\item Санкей;
		\item Графики с областями;
	\end{itemize}

	\item Статистические стили и стили Seaborn:
	\begin{itemize}
		\item Ошибки;
		\item Гистограммы;
		\item Диаграммы Facet и Trellis;
		\item Деревообразные графики;
		\item Графики-скрипки;
		\item Линия тренда.
	\end{itemize}

	\item Научные карты:
	\begin{itemize}
		\item Контур;
		\item Троичный сюжет;
		\item Логарифмический график;
		\item Поля векторов;
		\item Ковровый график;
		\item Радарчарт;
		\item Тепловые карты;
		\item Роза ветров;
		\item Полярный сюжет.
	\end{itemize}

	\item Финансовые графики;
	\item Карты;
	\item Подграфики;
	\item Трансформации;
	\item Взаимодействие Jupyter Widgets.
\end{enumerate}

\newpage
